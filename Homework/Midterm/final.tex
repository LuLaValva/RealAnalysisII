\documentclass{article}
\usepackage{amsthm}
\usepackage{amsmath}
\usepackage{amsfonts}
\usepackage{amssymb}
\usepackage[utf8]{inputenc}
\usepackage{titling}
\usepackage{scrextend}

\title{Real Analysis II - Homework II}
\author{Lucas LaValva}
\date{\today}

\usepackage{fancyhdr}
\pagestyle{fancy}
\lhead{\theauthor}
\rhead{\thetitle}

\begin{document}
\maketitle

\setcounter{section}{5}

\section{Series of Functions}

\subsection{Lim sup and Lim inf}
\begin{enumerate}
      \setcounter{enumi}{1}
      \item Let $a_n=1$ if $n=2^k$ for some positive integer $k$,
            and $a_n=\frac{1}{n!}$ otherwise.
            \begin{enumerate}
                  \item Find $\lim\sup a_n$ and $\lim\inf a_n$.
                  \item Find $\lim\sup\frac{\lvert a_{n+1}\rvert}{\lvert a_n\rvert}$.
                  \item Find $\lim\sup\lvert a_n\rvert^\frac{1}{n}$.
            \end{enumerate}
            \setcounter{enumi}{4}
      \item Let $\{a_n\}$ be a sequence of real numbers and suppose
            that $\lim\sup a_n$ is finite. Let $\{c_n\}$ be another
            sequence and suppose that $c_n\to c$.
            \begin{enumerate}
                  \item Prove that, if $c>0$, then
                        \begin{equation}
                              \lim\sup c_na_n = c\lim\sup a_n.
                              \label{eq:6.1_5_given}
                        \end{equation}
                  \item Find a counterexample to \eqref{eq:6.1_5_given}
                        with $c<0$.
            \end{enumerate}
            \setcounter{enumi}{8}
      \item Let $\{a_n\}$ be a bounded sequence of real numbers and
            let $P$ be the set of limit points of $\{a_n\}$. Limit
            points are defined in Section 2.6. Prove that
            $\lim\sup a_n=\sup P$ and $\lim\inf a_n=\inf P$.
\end{enumerate}

\subsection{Series of Real Constants}
\begin{enumerate}
      \item Determine whether the following series converge
            or diverge:
            \begin{enumerate}
                  \item $\sum\frac{3^j}{j!}$
                  \item $\sum\frac{\sqrt{n}}{1+n^2}$
                  \item $\sum\frac{2^j}{j^2}$
                  \item $\sum\frac{j^2}{j!}$
                  \item $\sum(\sqrt{j+1}-\sqrt{j})$
                  \item $\sum\frac{\sqrt{j+1}-\sqrt{j}}{j}$
                  \item $\sum e^{-j+\sin j}$
                  \item $\sum\frac{j+\cos j}{j}$
            \end{enumerate}
            \setcounter{enumi}{3}
      \item Consider the series $\sum_{j=1}^\infty\frac{1}{j(j+1)}$.
            \begin{enumerate}
                  \item Prove that the series converges by the
                        comparison test
                  \item What information does the ratio test give?
                  \item Use the formula $\frac{1}{j(j+1)}=\frac{1}{j}-\frac{1}{j+1}$
                        to compute the partial sums. Show explicitly that
                        the partial sums converge.
            \end{enumerate}
            \setcounter{enumi}{12}
      \item Prove that if $a_j\geq 0$ for all $j$ and $\sum a_j$
            converges, then $\sum a_j^2$ converges.
      \item Suppose that $a_j\geq 0$ and that $\sum a_j$ converges.
            \begin{enumerate}
                  \item Show by example that it is not necessarily true that
                        $\sum\sqrt{a_j}$ converges.
                  \item Show that $\sum\frac{\sqrt{a_j}}{j}$ converges.
            \end{enumerate}
\end{enumerate}

\subsection{The Weierstrass M-test}

\begin{enumerate}
      \setcounter{enumi}{2}
      \item \begin{enumerate}
                  \item Prove that $\sum_{j=0}^\infty x^j$ is differentiable
                        on $(0,1)$ and
                        \[
                              \frac{d}{dx}\sum_{j=0}^\infty x_j= \sum_{j=0}^\infty(j+1)x_j.
                        \]
                  \item Use the fact that $\sum_{j=0}^\infty x^j=\frac{1}{1+x}$ on
                        $(-1,1)$ to find a formula for $\sum_{j=0}^\infty(j+1)x^j$.
                  \item Use this to calculate $\sum_{j=0}^\infty\frac{j+1}{2^j}$ exactly.
            \end{enumerate}
            \setcounter{enumi}{4}
      \item \begin{enumerate}
                  \item Show that the series
                        \[
                              f(x)\equiv x+\sum_{j=1}^\infty x(1-x)^j
                        \]
                        converges for every $x$ in $[0,1]$.
                  \item What is $f(x)$?
                  \item Is the convergence uniform on $[0,1]$?
            \end{enumerate}
            \setcounter{enumi}{6}
      \item Let $f(x)$ be defined by
            \[
                  f(x)\equiv \sum_{j=1}^\infty\frac{1}{x^2+j^2}.
            \]
            \begin{enumerate}
                  \item Prove that $f$ is a well-defined, continuous function
                        on the whole real line.
                  \item Prove that $f$ is continuously differentiable and find a
                        series of representations for $f'$.
            \end{enumerate}
            \setcounter{enumi}{10}
      \item \begin{enumerate}
                  \item Show that the series
                        \[
                              f(x)\equiv \sum \frac{1}{j2^j}\sin jx
                        \]
                        converges uniformly on $[0,2\pi]$.
                  \item Prove that $f$ is uniformly continuous on $[0,2\pi]$.
                  \item Let $\varepsilon=10^{-3}$. Find a $\delta>0$ so that
                        $\lvert x-y\rvert\leq \delta$ implies that
                        $\lvert f(x)-f(y)\rvert\leq \varepsilon$.
            \end{enumerate}
\end{enumerate}

\subsection{Power Series}

\begin{enumerate}
      \setcounter{enumi}{1}
      \item Given the following conditions on the coefficients $\{a_j\}$, what
            can you say about the radius of convergence of $\sum a_jx^j$?
            \begin{enumerate}
                  \item $0<m_1\leq a_j\leq m_2$, for some constants $m_1$ and $m_2$.
                  \item $2^j\leq a_j\leq 3^j$.
                  \item $j^2\leq a_j\leq j^3$.
            \end{enumerate}
            \setcounter{enumi}{5}
      \item Using the power series, prove that
            $\frac{\tan x-x}{x^2}\to 0$ as $x\to 0$.
            \setcounter{enumi}{10}
      \item Find the radius of convergence of the series
            $\sum_{j=0}^\infty(j+1)(j+2)x^j$. Find the function to which
            the series converges.
\end{enumerate}

\subsection*{Projects (Chapter 6)}

\begin{enumerate}
      \item The purpose of this project is to develop a theory of
            \textbf{alternating series}. A series is called an alternating
            series if the signs of the terms alternate. For example,
            \[
                  1-\frac{1}{2}+\frac{1}{3}-\frac{1}{4}+\frac{1}{5}
                  -\frac{1}{6}+\frac{1}{7}-....
            \]
            is the alternating harmonic series. If the first term is positive,
            we can write the alternating series as $\sum_{j=0}^\infty(-1)^jc_j$,
            where $c_j\geq 0$ for all $j$. We will suppose that the sequence
            $\{c_j\}$ is non-increasing; that is, $c_j\geq c{j+1}$ for all $j$.
            \begin{enumerate}
                  \item Let $S_n=\sum_{j=0}^n(-1)^jc_j$. Prove that for each $n\geq 2$,
                        the number $S_n$ lies between $S_{n-1}$ and $S_{n-2}$.
                  \item  Use the Bolzano-Weierstrass theorem to show that a subsequence
                        $S_{n_k}$ converges.
                  \item Suppose, in addition, that $c_j\to 0$ as $j\to\infty$. Prove that
                        the whole sequence $\{S_n\}$ converges. This is known as the
                        \textbf{Alternating Series Theorem}.
                  \item Does the alternating harmonic series converge? Does it converge
                        absolutely?
                  \item Let $S=\lim_{n\to\infty}S_n$. Prove that
                        \[
                              \lvert S-S_n\rvert \leq c_{n+1}.
                        \]
                        Note that this gives us a very easy way to estimate how close a
                        partial sum is to the sum of an alternating series. How many terms
                        of the alternating geometric series do we have to take to be within
                        $10^{-4}$ of the limit?
                  \item Prove that for each $x$ we can choose a $J$ so that the Maclaurin
                        series for $\sin x$ and $\cos x$ satisfy the hypothesis for the
                        Alternating Series Theorem for $j\geq J$. Use the alternating series
                        remainder to estimate how close $1-\frac{x^2}{2!}+\frac{x^4}{4!}$ is
                        to $\cos x$ on the interval $[-1,1]$. Compare your estimate to one
                        you get by using Taylor's theorem.
            \end{enumerate}
\end{enumerate}


\setcounter{section}{8}
\section{Fourier Series}

\subsection{The Heat Equation}

\begin{enumerate}
      \setcounter{enumi}{3}
      \item Let $f(x)=x(L-x)$, as in Example 1, and suppose, for simplicity, that
            $L=\pi$ and $\kappa=1$. Let $S_n\equiv \sum_{j=1}^nb_ne^{-\lambda_nt}\sin nx$,
            where the $b_n$ are the same as those computed in Example 1.
            \begin{enumerate}
                  \item Generate the graphs of $f$, $S_1$, $S_3$, and $S_5$ on the interval
                        $[0,\pi]$ when $t=0$. Does it look as if the series for $u(0,x)$ is
                        converging to $f$?
                  \item Find an upper bound on the error we make if we replace $u$ by $S_5$.
                  \item Compare the graphs of $S_5$ for $t=0,.5,1,2,$ and $5$. Is this the
                        way you would expect $u(t,x)$ to behave?
            \end{enumerate}
      \item Let $L=\pi$ and $\kappa=1$ and define $f(x)=x$ on the interval $[0,\pi]$.
            \begin{enumerate}
                  \item Compute the coefficients $b_n$.
                  \item Show by explicit differentiation that thte function $u$ defined by
                        (10) satisfies the heat equation for $t>0$ and explain why the
                        boundary conditions (6) hold for all $t\geq 0$.
                  \item Explain why the series for $u$ at $t=0$ cannot converge to $f$ for
                        \textit{all} $x\in [0,\pi]$.
            \end{enumerate}
            \setcounter{enumi}{6}
      \item Consider the heat flow ina  bar where we do not prescribe the temperature
            at the ends but instead assume that the ends are insulated.
            \begin{enumerate}
                  \item Explain why the right boundary conditions are
                        \begin{equation}
                              u_x(t,0) = 0 = u_x(t,L).
                              \label{eq:9.1_7_given}
                        \end{equation}
                  \item Use the methods of the section to derive the formal solution
                        \[
                              u(t,x)=\sum_{n=0}^\infty a_ne^{-\lambda_n\kappa t}\cos\frac{n\pi x}{L},
                        \]
                        where $a_n=\frac{2}{L}\int_0^L f(x)\cos\frac{n\pi x}{L}dx$ and
                        $\lambda_n=(\frac{n\pi}{L})^2$, if $n>0$, and
                        $a_0=\frac{1}{L}\int_0^L f(x)dx$.
                  \item Explain why the same arguments as in Theorem 9.1.1 show that $u$
                        is infinitely differentiable in $x$ and $t$ for $t>0$ and $u$ satisfies
                        the heat equation and the boundary conditions \eqref{eq:9.1_7_given}.
                  \item By using the partial differential equation, prove that
                        $\int_0^Lu(t,x)dx$ is independent of $t$. Why is that reasonable?
                  \item What happens to the solution as $t\to\infty$?
            \end{enumerate}
      \item Suppose that the ends of the bar are kept at temperature zero and that
            $\beta cu(t,x)$ units of heat are added to the bar per gram per unit time
            by some internal chemical reaction where $\beta$ is constant. Show that $u$
            should satisfy the partial differential equation
            \begin{equation}
                  u_t(t,x)-\kappa u_{xx}(t,x)-\beta u(t,x) = 0.
                  \label{eq:9.1_8_given}
            \end{equation}
            Using the methods of the section, write down a formal solution to
            \eqref{eq:9.1_8_given} which satisfies the boundary condition (6) and the
            initial condition (5). How does the behavior of the solution as $t\to\infty$
            depend on $\beta$?
\end{enumerate}

\subsection{Definitions and Examples}

\begin{enumerate}
      \setcounter{enumi}{5}
      \item Assume that the Fourier series for $f(x)=(\pi-\lvert x\rvert)^2$,
            which we computed in Example 1, converges to $f(x)$ for each $x$ in
            $[-\pi,\pi]$. Prove that
            \[
                  \sum_{n=1}^\infty\frac{1}{n^2}=\frac{\pi^2}{6}.
            \]
            \setcounter{enumi}{7}
      \item Show that the Fourier series of the function $f(x)=\lvert x\rvert$ on
            the interval $[-\pi,\pi]$ is
            \[
                  f(x) \sim \frac{\pi}{2}-\frac{4}{\pi}
                  \sum_{n=1}^\infty\frac{1}{(2n-1)^2}\cos(2n-1)x.
            \]
            \setcounter{enumi}{9}
      \item Suppose that the Fourier coefficients of a piecewise continuous function
            $f$ satisfy $\sum m^k\lvert a_m\rvert<\infty$ and
            $\sum m^k\lvert b_m\rvert<\infty$ for some fixed positive integer $k$
            and that the Fourier series converges to $f(x)$ for all $x$. Prove that
            $f$ is $k$ times continuously differentiable.
\end{enumerate}

\subsection{Pointwise Convergence}

\begin{enumerate}
      \setcounter{enumi}{1}
      \item Let $f$ be the $2\pi$ periodic extension of the function
            \[
                  f(x) = \begin{cases}
                        2 + 3x & -\pi < x < 0   \\
                        x^2    & 0 \leq x < \pi \\
                        1      & x = \pi
                  \end{cases}
            \]
            \begin{enumerate}
                  \item For each $x\in\mathbb{R}$, find the right-hand and
                        left-hand limits of the difference quotient of $f$.
                  \item For each $x\in\mathbb{R}$, to what number does the
                        Fourier series of $f$ converge?
            \end{enumerate}
            \setcounter{enumi}{4}
      \item \begin{enumerate}
                  \item Compute the Fourier coefficients of the periodic extension
                        of the function $f(x)=x$ on $[-\pi,\pi)$.
                  \item Compute the graphs of $S_1$, $S_3$, and $S_{10}$ and compare
                        them to the graph of $f$.
                  \item Where does the Fourier series of $f$ converge? To what does
                        it converge?
                  \item Show that the solution of the heat equation discussed in
                        problem 5 of section 9.1 satisfies the initial condition
                        at all $x\in[0,\pi]$ except $x=\pi$.
            \end{enumerate}
            \setcounter{enumi}{6}
      \item Let $f$ be a twice continuously differentiable $2\pi$ periodic
            function. Prove that the Fourier series of $f$ converges to $f$
            uniformly.
            \setcounter{enumi}{9}
      \item A subset $E\subseteq \mathbb{R}$ is said to have \textbf{measure zero}
            if, given any $\varepsilon>0$, there is a countable family of intervals
            $\{I_n\}_{n=1}^\infty$ such that $E\subseteq\cup I_n$ and
            $\sum length(I_n)\leq \varepsilon$.
            \begin{enumerate}
                  \item Show that any finite set of points has measure zero.
                  \item Show that the set $\{\frac{1}{n}\}$ has measure zero.
                  \item Show that any countable set has measure zero. Remark:
                        There are uncountable sets of measure zero.
            \end{enumerate}
\end{enumerate}

\subsection{Mean-square Convergence}

\begin{enumerate}
      \setcounter{enumi}{1}
      \item Let $\{f_n\}$ be a sequence of continuous functions on $[a,b]$ that
            converges to a function $f$ uniformly. Prove that $f_n\to f$ in the
            mean-square sense.
      \item Let $[a,b]$ be a finite interval.
            \begin{enumerate}
                  \item Construct a sequence of continuous functions on $[a,b]$,
                        $\{f_n\}$, so that $f_n\to 0$ pointwise but
                        $\lVert f_n\rVert_2\to\infty$.
                  \item Construct a sequence of continuous functions on $[a,b]$,
                        $\{f_n\}$, so that $\lVert f_n\rVert_2\to\infty$ but
                        $\{f_n(x)\}$ does not converge to 0 for any $x\in[a,b]$.
            \end{enumerate}
            \setcounter{enumi}{5}
      \item \begin{enumerate}
                  \item Let $f$ be a piecewise continuous function on $[\pi,\pi]$.
                        Prove that there is a sequence of continuous functions
                        $f_n$ on $[\pi,\pi]$ so that $f_n\to f$ in mean-square sense.
                  \item Use the idea of the proof of theorem 9.4.6 to show that
                        the Fourier series of a piecewise continuous function $f$
                        converges to $f$ in the mean-square sense.
            \end{enumerate}
            \setcounter{enumi}{8}
      \item Let $f$ be a continuously differentiable function on $[-\pi,\pi]$
            such that \\$\int_{-\pi}^\pi f(x)dx=0$. Prove that
                  \begin{equation}
                        \int_{-\pi}^\pi \lvert f'(x)\rvert^2dx
                        \geq \int_{-\pi}^\pi \lvert f(x)\rvert^2dx.
                        \label{eq:9.4_9_given}
                  \end{equation}
                  Prove that strict equality holds in \eqref{eq:9.4_9_given} if and
                  only if $f(x)=a\cos x+b\sin x$ for some constants $a$ and $b$.
\end{enumerate}

\end{document}