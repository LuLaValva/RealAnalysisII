\documentclass{article}
\usepackage{amsthm}
\usepackage{amsmath}
\usepackage{amsfonts}
\usepackage[utf8]{inputenc}

\theoremstyle{definition}
\newtheorem{definition}{Definition}[section]

\title{Definitions and Theorems}
\author{Lucas LaValva}
\date{\today}

\begin{document}
\maketitle

%---------------------------------------------------
\section{Unit 0 --- Review}
\begin{definition}[Convergence]
    The sequence $a_n$ converges to $L$ when there exists
    N such that for all $n>\mathrm{N}$ and $\varepsilon>0$,
    $\left\lvert a_n-L\right\rvert\leq \varepsilon$.
\end{definition}
\begin{definition}[Continuity at point]
    A function $f(x)$ is continuous at point $p$ if, given
    $\varepsilon>0$, there exists $\delta>0$ such that if
    $\left\lvert p-x \right\rvert<\delta$ then
    $\left\lvert f(p)-f(x)\right\rvert<\varepsilon$.
\end{definition}
\begin{definition}[Uniform Continuity]
    A function $f(x)$ is uniformly continuous on domain
    $D\subset \mathbb{R}$ if, for $\varepsilon>0$, there
    exists $\delta>0$ such that for all $x,y\in  D$,
    $\left\lvert y-x\right\rvert<\delta \implies
        \left\lvert f(y)-f(x)\right\rvert <\varepsilon$.
\end{definition}
\begin{definition}[Differentiability]
    A function $f(x)$ is differentiable at point $p$ if,
    given $\varepsilon>0$, there exists $h>0$ such
    that $\left\lvert
        \frac{f(p+h)-f(p)}{h} - f'(x)
        \right\rvert<\varepsilon$.
\end{definition}
\begin{definition}[Cauchy Sequence]
    ${a_n}$ is a Cauchy sequence if there exists
    $N>0$ such that, given $\varepsilon>0$,
    $n, m > N \implies
        \left\lvert a_n-a_m\right\rvert <\varepsilon$.
\end{definition}

%---------------------------------------------------
\end{document}